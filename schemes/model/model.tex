\documentclass{standalone}
\usepackage[T2A,T1]{fontenc}
\usepackage[utf8]{inputenc}
\usepackage[russian]{babel}
\usepackage{tikz}
\usetikzlibrary {positioning}
\usetikzlibrary{shapes,arrows,decorations.pathmorphing,backgrounds,positioning,fit,petri}
\begin{document}
    \begin{tikzpicture}
        \node at ( 0,7) [rectangle,fill=blue!20,draw] (material) {\textbf{Свойства модельных материалов}};
        \node at ( 0,3) [rectangle,draw] (mainblock){%
            \begin{tikzpicture}
                \node at ( 0,2) [rectangle,fill=green!20,draw](KMC) {\textbf{Решеточный КМК}};
                \node at ( 5,2) [rectangle,fill=red!20,draw](charge) {\textbf{Распределение зарядов}};
                \node at ( 5,0) [rectangle,fill=green!20,draw](puasson) {\textbf{Решение 3d ур-ний Пуассона}};
                \node at ( 0,0) [rectangle,fill=red!20,draw](elnapr) {\textbf{Карта напряжений}};

                \draw [->] (KMC) -- (charge);
                \draw [->] (charge) -- (puasson);
                \draw [->] (puasson) -- (elnapr);
                \draw [->] (elnapr) -- (KMC);
            \end{tikzpicture}
        };
        \node at (-1, 1) [rectangle,fill=green!20,draw] (ekmk) {\textbf{КМК для тока}};
        \node at ( -3,0) [rectangle,fill=blue!20,draw] (current) {\textbf{Ток через устройство}};
        \node at ( 3,0) [rectangle,fill=blue!20,draw] (newstructure) {\textbf{Формирующаяся структура}};

        \node at ( -3,6) [rectangle,fill=blue!20,draw] (evaluation){\textbf{Сравнение с данными}};
        \node at ( 3,6) [rectangle,fill=blue!20,draw] (geometry){\textbf{Геометрия системы}};

        \draw [->] (material) -- (mainblock);
        \draw [->] (evaluation) -- (mainblock);
        \draw [->] (geometry) -- (mainblock);
        \draw [->] (mainblock) -- (ekmk);
        \draw [->] (ekmk) -- (current);
        \draw [->] (mainblock) -- (newstructure);
      \end{tikzpicture}
\end{document}